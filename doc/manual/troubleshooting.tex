\label{trouble}

\subsection{POP-Java exception}
This section lists some of the main POP-Java exception that can occurred during the application execution and gives the cause of the problem.

\subsubsection{Cannot bind to access point : socket://your-computer-name:2711}
This exception occurred when the application cannot contact the POP-C++ runtime system. To fix this problem, we need to start the POP-C++ runtime system with the following command : 

\begin{lstlisting}
POPC_LOCATION/sbin/SXXpopc start
\end{lstlisting}

\subsubsection{Error message : OBJECT\_EXECUTABLE\_NOTFOUND}
This exception occurred when the executable file is not found. This might be due to a bad object map or the deletion of the object executable file. To fix this problem we should generate a new object map with the object executable.

\subsubsection{Error message : NO\_RESOURCE\_MATCH}
This exception occurred when no resource match the requirements of a specific object. To fix this problem we should check the object descriptions in the parallel objects. We might have put a too high requirements for a parallel object creation.

\subsubsection{Error message : Cannot run program "/usr/local/popc/services/appservice"}
If we get an error with "cannot run program" and the path contains the appservice of POP-C++, sou have certainly reinstalled POP-C++ and the configuration file of POP-Java is now wrong. The easiest way to fix this problem is to reinstall also POP-Java. We can also edit the configuration file under \textit{POPJAVA\_LOCATION}/etc/popj\_config.xml. The item \textbf{popc\_appcoreservice\_location} must be modified with the good path.

\subsubsection{Test suite frozen}
If the test suite seems to be frozen, we should abort the test suite and restart the POP-C++ global service with the following command
\begin{lstlisting}
POPC_LOCATION/sbin/SXXpopc restart
\end{lstlisting}
